\documentclass[a4paper,10pt]{scrartcl}
\usepackage[utf8x]{inputenc}
\usepackage[T1]{fontenc}
\usepackage[usenames,dvipsnames]{xcolor}
\usepackage[utf8x]{inputenc}
\usepackage{algorithmicx}
\usepackage{algorithm}
\usepackage{algpseudocode}
\usepackage{amsmath,amssymb,amstext,mathabx}
\usepackage{amsthm,lipsum}
\usepackage{amsthm}
\usepackage{appendix}
\usepackage{caption}
\usepackage{diagbox}
\usepackage{enumitem}
\usepackage{float}
\usepackage{graphicx}
\usepackage{hyperref}
\usepackage{lmodern}
\usepackage{mathabx}
\usepackage{mathtools}
\usepackage{multicol}
\usepackage{parcolumns}
\usepackage{pdfpages}
\usepackage{pgfplotstable}
\usepackage{pstricks-add}
\usepackage{relsize}
\usepackage{subcaption}
\usepackage{tcolorbox}
\usepackage{verbatim}
%End of imports
%\colorlet{notgreen}{blue!50!yellow}
\definecolor{orange}{rgb}{1,0.5,0}
\definecolor{jwbluedark}{HTML}{4f81bd}
\definecolor{jwbluelight}{HTML}{dbe5f1}
\newcommand{\mybox}[1]{\par\noindent\colorbox{jwbluelight}{\parbox{\dimexpr\textwidth-2\fboxsep\relax}{#1}}}
\newcommand{\TODO}[1]{\ifture\colorbox{red}{TODO} \fi}
\newcommand{\R}{\mathbb{R}}
\newcommand{\C}{\mathbb{C}}
\newcommand{\N}{\mathbb{N}}
\newcommand{\del}{\partial}
\newcommand{\la}{\langle}
\newcommand{\ra}{\rangle}
\newcommand{\rarr}{\rightarrow}

\makeatletter
\def\thm@space@setup{%
    \thm@preskip=1.5cm
  \thm@postskip=\thm@preskip % or whatever, if you don't want them to be equal
}
\makeatother
%\floatname{algorithm}{Algorithmus}
%\floatstyle{plain}
%\renewcommand*\listalgorithmname{ }
%\restylefloat{table}
%Definiere Saetze ung Definitionen:
\newtheorem{defi}{Definition}[section]
\newtheorem{gleichung}{Equation}[section]
\newtheorem{satz}{Theorem}[section]
\newtheorem{corr}{Corollary}[section]
\newtheorem{bsp}{Example}[section]
\newtheorem{lemma}{Lemma}[section]
\newtheorem{propo}{Theorem}[section]
\newtheorem{bem}{Note}[section]
%Titel Author
\author{Christoph Höppke}
\date{\today}
%\newcommand{\pasteimg}[2]{}


%CustomColors:
\definecolor{Syellow}{HTML}{B58900} \definecolor{Sorange}{HTML}{CB4b16}
\definecolor{Sred}{HTML}{DC322F} \definecolor{Smagenta}{HTML}{D33682}
\definecolor{Sviolet}{HTML}{6C71C4} \definecolor{Sblue}{HTML}{268BD2}
\definecolor{Scyan}{HTML}{2AA198} \definecolor{Sgreen}{HTML}{859900}


\begin{document}
\pagestyle{empty}
\begin{center} \vspace{4ex}
    \rule{\textwidth}{1.6pt}\vspace*{-\baselineskip}\vspace*{2pt} % Thick horizontal line
    \rule{\textwidth}{0.4pt}\\[\baselineskip]  % Thin horizontal
    \large{Oxford, \today}\\ \vspace{2ex}
    \textbf{\Large{Induction Talks}}\\
    \rule{\textwidth}{0.4pt}\vspace*{-\baselineskip}\vspace{3.2pt} % Thin horizontal line
    \rule{\textwidth}{1.6pt}\\[\baselineskip] % Thick horizontal
    \vspace{2ex}
\end{center}
\vfill
\begin{flushright}
    \begin{minipage}[h]{0.3\textwidth}
        \begin{normalsize}
            Christoph M. Hoeppke\\
            D.Phil Mathematics\\
            InFoMM CDT\\
            University of Oxford\\
            christoph.hoeppke@maths.ox.ac.uk
        \end{normalsize}
    \end{minipage}
\end{flushright}
\pagebreak
%\maketitle \newpage
\newpage
\tableofcontents
\newpage
\pagestyle{plain}
\setcounter{page}{1}
%\setcounter{section}{-1}
\section{Why use Version Control}
Version control is better than mailing files back and forth.
Nothing that is committed is lost and you can track changes.

\subsection{What can it be used for}%
\label{sub:what_can_it_be_used_for}
Use git for:
\begin{itemize}
    \item Code
    \item Papers
    \item Presentations
    \item Grant applications
\end{itemize}

Git is the latest and greatest version control software. A lot of people are
starting to use Git / Version control.
Developt in 2005 by the Linux development community for the Linux kernel
project.
Features:
\begin{itemize}
    \item Branching and merging
    \item Fast
    \item Distributed (This distinguishes git from its main competitor
        Subversion)
    \item Flexible staging area
    \item Free and open source
    \item Good tutorials
\end{itemize}

From the readme of the source code:
The name git was given by Linus Torvalds.

\section{What is a repository}%
\label{sec:what_is_a_repository}

A repository is a series of commits.
Each commit contains snapshots of the files that are added along with timing,
author and other information. As well as the commits, branch pointers show the
progress of different branches. The working directory is simply the current set
of files in the user's local directory. The staging area is where new edits are
added in preparation for creating new commit. The working directory is just the
content of a directory on the current users hard disk.

\subsection{Creating a Repository}%
\label{sub:creating_a_repository}

Normally you wold work with a remote repository. you can import this into your
current directory using the clone command.\\
\begin{indent}
    git clone <remote repository> <local dir>
\end{indent}\\

Use git add to add a change in the working directory to the staging ares. All
changes or new files need to be added to the staging area before they can be
committed.

\indent git add <file>\\
\indent git add <directory>

\subsection{Sending commits to a Repository}%
\label{sub:sending_commits_to_a_repository}
\indent git push origin <branch>

\subsection{Examining the history}%
\label{sub:examining_the_history}
\indent git log --online\\
\indent git checkout <commit>
The git checkout command sets the head pointer the given commit pointer.\\
\indent git checkout master\\
To exit the status where you are not pointing at a branch head use the git
checkout master command.\\
\indent git status\\
To see the current status of git.

\subsection{Branching}%
\label{sub:branching}
Branching can be used to:
\begin{itemize}
    \item Create / try out a new feature
    \item Provide conflict-free parallel editing for multiple users.
\end{itemize}

How to create a branch:\\
\indent git checkout -b <branch>\\
\indent git commit -asm ``added wow new featre''

\subsection{Merging}%
\label{sub:merging}

When we have two separate branches, but what if we want to merge them
together?\\
\indent git chekcout master\\
\indent git merge branch\\
In the case that there is a merge conflict, git will let you handle the merge
conflict. Once you are done you may commit the merged result.

\section{Additional resources}%
\label{sec:additional_resources}

\begin{itemize}
    \item Git book: Createive commons Attribution-NonCommercial software.
\end{itemize}



\indent git commit -asm ``merges branch into master''



\end{document}
